% Source: http://tex.stackexchange.com/a/5374/23931
\documentclass[11pt]{article}
\usepackage[T1]{fontenc}
\usepackage[utf8]{inputenc}
\usepackage[margin=1in]{geometry}

\newcommand{\HRule}{\rule{\linewidth}{0.7mm}}
\newcommand{\Hrule}{\rule{\linewidth}{0.5mm}}
\setlength{\parindent}{0pt}
\makeatletter% since there's an at-sign (@) in the command name
\renewcommand{\@maketitle}{%
  \parindent=0pt% don't indent paragraphs in the title block
  \centering
  {\Large \bfseries\textsc{\@title}}
  \HRule\par%
  \textit{\@author \hfill \@date}
  \par
}
\makeatother% resets the meaning of the at-sign (@)

\title{Machine-1 EMS}
\author{Devesh Tarasia}
\date{19-Jan-2021}

\begin{document}
  \maketitle% prints the title block
  \vspace{0.2cm}
  \textbf{Problem Statement:}\newline
  To create a program that would be able to monitor the status of machine-1 and be able to detect the type of 
  fault, the part associate with the fault and possible rectification\newline

  
  \textbf{Type of Data:}
  \vspace{-2mm}
  \begin{itemize}
    \setlength\itemsep{0em}
    \item Functional Nodes: The values of these nodes constantly change throughout the operation of the machine 
    and and can be used as an indicator of progress of the process through the workstation
    \item State Nodes: The values of these nodes remain unchanged throughout the operation and are an indicator
    of the 'health' of the machine. If some state tags are wrong, then the machine is not in the right state to
    operate
    \begin{itemize}
      \itemsep0em
      \item State Nodes common for all components
      \item State Nodes specific for each component
    \end{itemize}
    \item Operational Times: Timestamps of the Functional Nodes when a change of value occurs in the nodes
  \end{itemize}
  \textbf{Current Approach:}\newline
  \textit{Idea:}\newline
  The idea is to compare the node values at each event change with the correct values. The correct values are stored in a matrix
  where each row has the values of functional nodes at each change.\newline
  At each event change, the values of all the functional nodes is checked with corresponding row. The row numbers represent the
  the event number, which are incremented chronologically.\newline
  To check the difference between 
\end{document}
